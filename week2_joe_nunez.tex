\documentclass[12pt]{article}
\usepackage{url,graphicx,tabularx,array,geometry,amsmath,amssymb,amsthm,lipsum,hyperref}
\setlength{\parskip}{1ex} %--skip lines between paragraphs
\setlength{\parindent}{0pt} %--don't indent paragraphs

%-- Commands for header
\renewcommand{\title}[1]{\textbf{#1}\\}
\renewcommand{\line}{\begin{tabularx}{\textwidth}{X>{\raggedleft}X}\hline\\\end{tabularx}\\[-0.5cm]}
\newcommand{\leftright}[2]{\begin{tabularx}{\textwidth}{X>{\raggedleft}X}#1%
& #2\\\end{tabularx}\\[-0.5cm]}

\newcommand{\unionf}{\bigcup_{n=1}^{\infty} \mathcal{F}_n}
\newcommand{\F}{\mathcal{F}}
\newcommand{\E}{\mathbb{E}}
\newcommand{\PP}{\mathbb{P}}
\newcommand{\R}{\mathbb{R}}
\newcommand{\C}{\mathbb{C}}
\newcommand{\A}{\mathbb{A}}
\newcommand{\Z}{\mathbb{Z}}
\newcommand{\Rad}{\text{Rad}}

\newcommand{\Hi}{\mathcal{H}}
\newcommand{\Aseq}{\{A_k\}_{k=1}^{\infty}}
\newcommand{\unionA}{\bigcup_{k=1}^{\infty} A_k}
\newcommand{\argmax}{\operatornamewithlimits{argmax}}

%\linespread{2} %-- Uncomment for Double Space
\begin{document}

\title{Math 176 F18: Week 2}
\line
\leftright{Due: Sept. 17, 2018}{Nunez, Joseph, \texttt{jbnunez@hmc.edu}} %-- left and right positions in the header



\begin{enumerate}
 
 	\item \textbf{4.2.13} Show that $I(V)$ is an ideal in the ring $k[x_1,\dots, x_n]$. (Class)
 	
 	

    \item \textbf{4.2.14(2)} Let $X$ be a set of points in $\A^n(\C)$.  Find a set $X$ with $X\neq V(I(X))$
    
    Let $X=\{ x\in \A^n(\C): x_1 \in \R, x_i = 0 \forall i>1\}$, the subset of $\A^n(\C)$ with only real first index and zero everywhere else.  Since the reals are not an algebraic set in the complex numbers (shown in 4.1.5.3), the zero set of the ideal they generate cannot be the reals.  
    
    \item \textbf{4.2.15}  Let $I$ be an ideal in $k[x_1,\dots, x_n]$. \begin{enumerate}
    \item Show that $I \subseteq I(V(I))$.
    
    Consider an ideal $I$.  Take $P\in I$.  
    Then $V(I) = \{a_1,\dots,a_n) \in \A^n(k): P(a_1,\dots,a_n)=0 ~\forall P\in I\}$ and $I(V(I) = \{P \in k[x_1,\dots, x_n]: P(a_1,\dots,a_n)=0 ~\forall (a_1,\dots,a_n)\in V(I)\}$.  By construction, $P(a_1,\dots,a_n)=0$ for all $(a_1,\dots,a_n)\in V(I)$.  Therefore $P\in I(V(I))$.  Thus $I \subseteq I(V(I))$.
    
    \item Find an ideal $I$ with $I\neq I(V(I))$.
    
    Consider the ideal generated by $\{x^2, x^3\}$ in $\R[x]$.  The ideal $I$ generated is all polynomials who have no instance of $x$ to the first or zeroeth power, but all other integer powers are present.  Then $V(I)=\{0\}$.  The ideal $I(\{0\})$ includes the polynomial $x$, since it vanishes on $\{0\}$.  Since $x$ is not in $I$, $I \neq I(V(I))$.
    
    \item Show that if $I$ is the ideal of an algebraic set, then $I=I(V(I))$.
    
    Let $I$ be the ideal of some algebraic set $X$.  Then $I$ vanishes on exactly $X$ and no other values in the field.  Therefore $V(I) = X$.  Then $I(V(I))$ is the ideal of the algebraic set $X$, which is simply $I$.  Thus $I=I(V(I))$.
    
    \end{enumerate}
    
    \textbf{Defn: Radical Ideal} The radical of an ideal $I$ is defined as
    \[\Rad(I) = \{P \in k[x_1,\dots,x_n] : P^m \in I, m>0\}\]
    If $I=\Rad (I)$, then $I$ is a radical ideal.  Also denoted as $\sqrt{I}$
    
    \item \textbf{4.2.18}  Let $X$ be a set of points in $\A^n(k)$.  Show that $I(X)$ is a radical ideal.
      
      $I(X)$ is the set of all polynomials that vanish on $X$.  Now suppose, to obtain a contradiction, that there exists some $P$ such that $P\not\in I(X)$ but $P^m \in I(X)$ for some $m>0$.  Since $P\not\in I(X)$ then for some $x\in X$, $P(x) = a$ for some nonzero $a\in \A^n(k)$.  But then $P^m(x) = (P(x))^m = a^m = 0$.  However, since we are in a field $k$, $a^m$ cannot equal zero since fields do not have zero divisors, so we have a contradiction.  Thus $I(X)$ is a radical ideal.
      
    \item \textbf{4.2.19}  Show that $\Rad(I)\subset I(V(I))$ for any ideal $I$ in $k[x_1,\dots, x_n]$.
      
      Let $P\in \Rad(I)$.  Then $P^m \in I$ for some $m>0$.  Let $x\in V(I)$.  Then $P^m(x) = 0$, so $(P(x))^m = 0$.  Since we are in a field, this implies $P(x)=0$.  Therefore $P$ vanishes on $V(I)$.  Thus $P\in I(V(I))$.  Therefore $\Rad(I)\subset I(V(I))$.
     
    \item \textbf{4.2.20}  Let $I$ be an ideal in $k[x_1,\dots, x_n]$.  Show $V(I)=V(\Rad(I))$.
      
    Let $x \in V(I)$.  Then for all $P\in I$, $P(x)=0$.  Let $P\in \Rad(I)$.  Then $P^m \in I$, so $P^m(x) = 0$.  
    Since we are in a field, this implies $P(x)=0$. Therefore $P(x) = 0$ for all $P\in \Rad(I)$, so $x\in V(\Rad(I))$.  
    Thus $V(I)\subseteq V(\Rad(I))$.
    
    Let $x \in V(\Rad(I))$.  Then for all $P\in \Rad(I)$, $P(x)=0$.  Let $P\in I$.  Then $P = Q^m$ for some $Q \in \Rad(I)$.  Since $Q\in \Rad(I)$, $Q(x)=0$ so $P(x) = 0^m = 0$.  Therefore $x\in V(I)$.  Thus $V(\Rad(I)) \subseteq V(I)$.
    
    By dual containment, $V(I)=V(\Rad(I))$.  
     
    %\item \textbf{4.3.2} Show that every field and principal ideal domain (PID) is Noetherian.  (Recall that a ring is a PID if whenever $xy=0$, then $x=0$ or $y=0$, and ever nontrivial ring is generated by a single element.)
        \item \textbf{4.2.21} Let $X$ and $W$ be algebraic sets in $\A^n(k)$.  Show that $X\subset W$ if and only if $I(X)\supset I(W)$.  Conclude that $X= W$ if and only if $I(X)= I(W)$. 
        
        Forwards: Suppose that $X\subset W$.  Take $p\in I(W)$.  Then $\forall x\in X, x\in W \implies p(x)=0 \implies p(s)\in I(X) \implies I(W)\subset I(X)$.
        
        Reverse: Suppose $I(X)\supset I(W)$ and $X \subset W$ are both algebraic, hence there exist $P_W, P_X$ such that $V(P_W)=W, V(P_X)=X$.  Take $y\in X$.  For every $p\in P_W, \forall w\in W, p(w)=0$.  Then $p\in I(W) \implies p\in I(X)$.  Thus $\forall y\in X, p(y)=0 \implies y\in V(P) \implies X\subset W$.
     
    \item \textbf{4.3.3} Let $R$ be a ring.  Prove that the following three conditions are equivalent:
    \begin{enumerate}
    \item $R$ is Noetherian.
    
    \item Every ascending chain $I_1 \subseteq I_2 \subseteq \cdots \subseteq I_n \subseteq \cdots$ of ideal in $R$ is stationary, i.e. there exists $N$ such that for $n\geq N$, $I_n=I_N$.  This is called the ascending chain condition (ACC).
    
    \item Every nonempty set of ideals in $R$ has a maximal element (with inclusion being being the ordering between ideals).  This means that if we have a set of ideals $\{I_1,I_2,\dots\}$, there must be at least one ideal, say $I_k$, such that there is no $I_n$ in the set with 
    \[i_k \subsetneq I_n.\]
    (There can be more than one maximal element.)
    \end{enumerate}
    
    (a) implies (b): Let $R$ be Noetherian.  Consider an ascending chain of ideals $J = I_1 \subseteq I_2 \subseteq \cdots \subseteq I_n \subseteq \cdots$ in $R$.  Let $I = \bigcup_{i\in J} I_i$.  Note that this $I$ is an ideal since it is no larger than the largest set in the union, so it is an ideal.  Since all ideals are finitely generated, $I$ must also be finitely generated by some finite $n$ generators.  Thus each of these generators must be in some $I_i$.  Consider the generator which appears last in the chain, say in $I_m$.  By the ascending chain of containment, $I_m$ must also contain the other generators, so $I_m=I$, so the chain is stable.
        
    (b) implies (c): Suppose every ascending chain of ideals is stationary.  Consider a set of ideals $I=\{I_\alpha\}_\alpha$.  Pick some $I_1$ from $I$.  Now take $I^1= I- \{I_\alpha\}_\alpha$.  Now pick $I_2$ from $I^1$ such that $I_1\subseteq I_2$.  Repeat this process until it is not possible to pick a $I_{n+1} \supseteq I_n$, then $I_n$ is maximal.  If this process continues for an arbitrarily long time, then we have an ascending chain of ideals, in which case that chain must be stationary by our initial assumption.
    
    (c) implies (a): Let $I$ be an ideal in $R$, and let $S$ be the set of all finitely generated ideals contained in $I$. $S\neq \emptyset$ because it contains the ideal of generated by $0$.  Let $J$ be the maximal element of $S$.  Suppose to obtain a contradiction that $J\neq I$.  Then there exists $a\in I$ such that $a\not\in J$.  Then $J$ is contained in the ideal generated by $a$ and $J$.  This contradicts maximality, so $I$ is finitely generated
    
     \item \textbf{4.3.4} (Hilbert Basis Theorem) If $R$ is a Noetherian ring, then $R[x]$ is also a Noetherian ring.  
     
     Proof:
     Let $I\subset R[x]$ be an ideal of $R[x]$.  We will show $I$ to be finitely generated.  
     
     \textbf{Step 1}: Let $f_1$ be a nonzero element of least degree in $I$.  For $i>1$, let $f_i$ be an element of least degree in $I - \langle f_1, \dots, f_{i-1} \rangle$, if possible.  If not possible, then $I=\langle f_1, \dots, f_{i-1} \rangle$, so it is finitely generated.
     
     \textbf{Step 2}: For each $i$, write $f_i = a_i x^{d_1} +$ lower order terms.  That is, let $a_i$ be the leading coefficient of $f_i$.  Set $J=\langle a_1, a_2, \dots \rangle$.  
     
     \textbf{Step 3}: Since $R$ is Noetherian, $J$ must be finitely generated such that $J=\langle a_1, \dots, a_m \rangle$ for some $m$.
     
     \textbf{Step 4}: Since $\R$ is Noetherian, $J=\langle a_1, \dots, a_m\rangle$ for some $m$.
     
	\item \textbf{4.3.5} Justify Step 4.
	
	Let $J=\langle a_1, a_2, \dots\rangle$.  Then $\langle a_1 \rangle \subseteq \langle a_1,a_2 \rangle \subseteq \dots$ forms an ascending chain, which is stationary by 4.3.3.  Thus $J=\langle a_1, \dots, a_m\rangle$ for some $m$.
     
     \textbf{Step 5}: Claim that $I=\langle f_1, \dots, f_m\rangle$.  If not, there is an $f_{m+1}\in I$, and we can subtract off its leading term using elements of $\langle f_1, \dots, f_m\rangle$ to get a contradiction.  
     
     \item \textbf{4.3.6} Fill in the details of Step 5.
    
    Suppose there exists $f_{m+1}\in \langle f_1, \dots, f_m\rangle$.  
    Then $f_{m+1} = a_{m+1} x^{ d_{m+1} }$ plus lower order terms.
     $a_{m+1}\in J$, so $a_{m+1} x^{ d_{m+1} } = \sum_{i=1}^m c_i f_i x^i$ plus lower order terms, contradicting minimality of $m$.
      
      \item \textbf{4.3.9} Let $f_1, \dots, f_k \in \R[x_1, \dots, x_n]$ and $V=V(f_1, \dots, f_k) \subset \A^k(\R)$.  Show that there is a polynomial $f\in \R[x_1, \dots, x_n]$ such that $V(f_1, \dots, f_k) = V(f)$.  Give an example to show that this fails over $\C$.  (Class)
      
      Claim: $F = \sum_{i=1}^k (f_i)^2$ satisfies $V(f_1, \dots, f_k) = V(f)$. 
      
      Forwards: Take $x\in V(f_1, \dots, f_k)$.  Then $f_i(x)=0$ for $i=1,\dots, k$.  Then $f = \sum_{i=1}^k 0 = 0$, so $x\in V(f)$.  Thus $x\in V(f_1, \dots, f_k)\subset V(f)$,
      
      Reverse: Take $x\in V(f)$.  Then $f(x)= \sum_{i=1}^k (f_i(x))^2 = 0$.  Since all terms are in squares in $\R$, all terms in the sum are nonnegative.  Therefore, for the sum to be zero, all of the terms must be zero.  Thus $f_i(x)=0$ for $i=1,\dots, k$. Thus $x\in V(f_1, \dots, f_k)$, so $V(f)\subset V(f_1, \dots, f_k)$.
      
      By dual containment, $V(f_1, \dots, f_k)=V(f)$.
      
      Counterexample: Consider $f_1=x, f_2=y$ in $\C[x,y]$.  $V(f_1, f_2)=\{(0,0)\}\in \A^2(\C)$.  Suppose there exists some $p\in \C[x,y]$ such that $V(p) = \{(0,0)\}$, and let $D$ be the degree of $p$.  Also consider $p'\in \C[x]$, which we get by sending $y\mapsto x$ in $p$, i.e $p'(x) = p(x,x)$.  Note (1) that the degree of $p'$ must be the same as the degree of $p$, and also that if we have ??? nvm this proof didn't work.  
      
      *Prof. Karp takes over and gives lecture*
      
      $\{x\subset \A^n(k) \leftarrow_V \rightarrow_I \{\text{Ideals of }k[x_1, \dots, x_n]\}$.  This is not a direct correspondence, since not all subsets of the affine space are algebraic, and not all ideals are radical.  We do have a bijection
      \[\{\text{Algebraic subset of }\A^n(k)\} \leftrightarrow \{\text{Radical Ideals}\}\]
      This bijection will be the focus of future classes.
   
\end{enumerate}

% HIDDEN SECTIONS
% =============================================================================================================
% This section gives you some things you can copy and paste for when you need lists, figure, and math notation.


% Figure with caption and label (labels are for referencing):
% -----------------------------------------------------------
% \begin{figure}[htbp]
%   \centering
%   \caption{Source-Channel Diagram: (P) part-of-speech, (T) word token.}
%   \includegraphics[width=.5\linewidth]{images/hw_1_1B.pdf}
%   \label{fig:source-channel-B}
% \end{figure}


% Enumerated list:
% -----------------        
% \begin{enumerate}
%    \item Item 1
%    \item 2
% \end{enumerate}

% Multiline equation with numbers:
% ------------------------------------
%    \begin{align}
%        C^* &= \argmax_{C} \mathbb{P}(C|G) \\
%            &= \argmax_{C} \mathbb{P}(C)\mathbb{P}(G|C).
%    \end{align}

% Multiline equation without numbers:
% ------------------------------------
%    \begin{align*}
%        C^* &= \argmax_{C} \mathbb{P}(C|G) \\
%            &= \argmax_{C} \mathbb{P}(C)\mathbb{P}(G|C).
%    \end{align*}        
 
% Math notation:
% --------------
% Inline equations are done like $\textbox{this} E = MC^2$. If you want an unnumbered equation on its own line, use double dollar signs $$x \in \mathcal{X} \bigcup \mathcal{Y}$$.
%
% \[ 
%   \E[\log_2 X] = \int_{a}^{b} p(x) \log_2 x dx.
% \]
%
%

% REFERENCES (use BibTeX, with a file named references.bib, for example. Just Google BibTex to see how to use it.):
% --------------
% \bibliographystyle{plain}
% \bibliography{references}

\end{document}
